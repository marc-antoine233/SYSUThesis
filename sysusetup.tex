% !TeX root = ./main.tex

\sysusetup{
  title                = {中山大学毕业论文/设计 \LaTeX\ 模板(试用)},
  title*               = {\LaTeX\ template for thesis/dissertation of Sun Yat-sen University (draft)},
  author               = {法核哥},
  author*              = {1FCENdoge},
  speciality           = {核工程与核技术},
  speciality*          = {Nuclear Engineering and Technology},
  supervisor           = {XXX~副教授},
  supervisor*          = {Assoc. Prof. XXX},
  % practice-supervisor  = {XXX~教授, XXX~教授},
  % practice-supervisor* = {Prof. XXX, Prof. XXX},
  % date                 = {2017-05-01},  % 完成时间,默认为今日
  % professional-type    = {专业学位类型},
  % professional-type*   = {Professional degree type},
   department           = {中法核工程与技术学院},  % 院系,本科生需要填写
   student-id           = {15356000},  % 学号,本科生需要填写
  % secret-level         = {秘密},     % 绝密|机密|秘密|控阅,注释本行则公开
  % secret-level*        = {Secret},  % Top secret | Highly secret | Secret
  % secret-year          = {10},      % 保密/控阅期限
  % reviewer             = true,      % 声明页显示“评审专家签名”
  %
  % 数学字体
  % math-style           = GB,  % 可选:GB, TeX, ISO
  math-font            = xits,  % 可选:stix, xits, libertinus
  cite-style           = super,
}

%\renewcommand{\thefigure}{\thechapter-\arabic{figure}}% 将正文和目录的图片标题由x.x改为x-x,慎用
%\renewcommand{\thetable}{\arabic{chapter}-\arabic{table}}% 将正文和目录的表格标题由x.x改为x-x,慎用
%\renewcommand{\theequation}{\arabic{chapter}-\arabic{equation}}% 将正文的公式序号由(x.x)改为(x-x),慎用

% 加载宏包

% 定理类环境宏包
\usepackage{amsthm}

% 表注
\usepackage{threeparttable}

% 跨页表格
\usepackage{longtable}

% 算法
\usepackage[ruled,linesnumbered]{algorithm2e}
%\renewcommand{\thealgocf}{\arabic{chapter}-\arabic{algocf}}% 将算法标题由x.x改为x-x,慎用

% SI 量和单位
\usepackage{siunitx}

% 参考文献使用 BibTeX + natbib 宏包
% 顺序编码制
\usepackage[sort]{natbib}
\bibliographystyle{sysuthesis-numerical}

% 配置图片的默认目录
\graphicspath{{figures/}}

% 数学命令
\makeatletter
\newcommand\dif{%  % 微分符号
  \mathop{}\!%
  \ifsysu@math@style@TeX
    d%
  \else
    \mathrm{d}%
  \fi
}
\makeatother
\newcommand\eu{{\symup{e}}}
\newcommand\iu{{\symup{i}}}

% 用于写文档的命令
\DeclareRobustCommand\cs[1]{\texttt{\char`\\#1}}
\DeclareRobustCommand\env[1]{\texttt{#1}}
\DeclareRobustCommand\pkg[1]{\textsf{#1}}
\DeclareRobustCommand\file[1]{\nolinkurl{#1}}

% hyperref 宏包在最后调用
\usepackage{hyperref}
